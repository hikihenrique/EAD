\documentclass[
	% -- opções da classe memoir --
	12pt,				% tamanho da fonte
	%openright,			% capítulos começam em pág ímpar (insere página vazia caso preciso)
	oneside,			% para impressão em verso e anverso. Oposto a oneside
	a4paper,			% tamanho do papel. 
	% -- opções da classe abntex2 --
	%chapter=TITLE,		% títulos de capítulos convertidos em letras maiúsculas
	%section=TITLE,		% títulos de seções convertidos em letras maiúsculas
	%subsection=TITLE,	% títulos de subseções convertidos em letras maiúsculas
	%subsubsection=TITLE,% títulos de subsubseções convertidos em letras maiúsculas
	% -- opções do pacote babel --
	english,			% idioma adicional para hifenização
	french,				% idioma adicional para hifenização
	spanish,			% idioma adicional para hifenização
	brazil,				% o último idioma é o principal do documento
	]{abntex2}

% ---
% PACOTES
% ---

% ---
% Pacotes fundamentais 
% ---
\usepackage{lmodern}			% Usa a fonte Latin Modern
\usepackage[T1]{fontenc}		% Selecao de codigos de fonte.
\usepackage[utf8]{inputenc}		% Codificacao do documento (conversão automática dos acentos)
\usepackage{indentfirst}		% Indenta o primeiro parágrafo de cada seção.
\usepackage{color}				% Controle das cores
\usepackage{graphicx}			% Inclusão de gráficos
\usepackage{microtype} 			% para melhorias de justificação
% ---

% ---
% Pacotes adicionais, usados apenas no âmbito do Modelo Canônico do abnteX2
% ---
\usepackage{lipsum}				% para geração de dummy text
% ---

% ---
% Pacotes de citações
% ---
\usepackage[brazilian,hyperpageref]{backref}	 % Paginas com as citações na bibl
\usepackage[alf]{abntex2cite}	% Citações padrão ABNT

% --- 
% CONFIGURAÇÕES DE PACOTES
% --- 

% ---
% Configurações do pacote backref
% Usado sem a opção hyperpageref de backref
\renewcommand{\backrefpagesname}{Citado na(s) página(s):~}
% Texto padrão antes do número das páginas
\renewcommand{\backref}{}
% Define os textos da citação
\renewcommand*{\backrefalt}[4]{
	\ifcase #1 %
		Nenhuma citação no texto.%
	\or
		Citado na página #2.%
	\else
		Citado #1 vezes nas páginas #2.%
	\fi}%
% ---

% ---
% Informações de dados para CAPA e FOLHA DE ROSTO
% ---
\titulo{Motivo de haver falta de mulheres na área de Técnologia da Informaç\~{a}o}
\author{Caio de Lazari Rosa,\\ Henrique Shodi Maeta e Matheus Rodrigues Souza}
\local{S\~{a}o~Paulo, SP}
\data{Abril de 2016}
\instituicao{%
  Centro Universitário Senac, Santo Amaro
  \par
  Faculdade de Ciência da Computaç\~{a}o}
\tipotrabalho{Projeto de Pesquisa}
% O preambulo deve conter o tipo do trabalho, o objetivo, 
% o nome da instituição e a área de concentração 
\preambulo{O objetivo de nossa pesquisa é entender e explicar o motivo pela qual há a falta de mulheres especializadas na área de Técnologia da Informaç\~{a}o.
Orientadora: Angela Pintor dos Reis}
% ---

% ---
% Configurações de aparência do PDF final

% alterando o aspecto da cor azul
\definecolor{blue}{RGB}{41,5,195}

% informações do PDF
\makeatletter
\hypersetup{
     	%pagebackref=true,
		pdftitle={\@title}, 
		pdfauthor={\@author},
    	pdfsubject={\imprimirpreambulo},
	    pdfcreator={LaTeX with abnTeX2},
		pdfkeywords={abnt}{latex}{abntex}{abntex2}{projeto de pesquisa}, 
		colorlinks=true,       		% false: boxed links; true: colored links
    	linkcolor=blue,          	% color of internal links
    	citecolor=blue,        		% color of links to bibliography
    	filecolor=magenta,      		% color of file links
		urlcolor=blue,
		bookmarksdepth=4
}
\makeatother
% --- 

% --- 
% Espaçamentos entre linhas e parágrafos 
% --- 

% O tamanho do parágrafo é dado por:
\setlength{\parindent}{1.3cm}

% Controle do espaçamento entre um parágrafo e outro:
\setlength{\parskip}{0.2cm}  % tente também \onelineskip

% ---
% compila o indice
% ---
\makeindex
% ---

% ----
% Início do documento
% ----
\begin{document}

% Retira espaço extra obsoleto entre as frases.
\frenchspacing 

% ----------------------------------------------------------
% ELEMENTOS PRÉ-TEXTUAIS
% ----------------------------------------------------------
% \pretextual

% ---
% Capa
% ---
%\imprimircapa
% ---

% ---
% Folha de rosto
% ---
\imprimirfolhaderosto
% ---

% ---
% NOTA DA ABNT NBR 15287:2011, p. 4:
%  ``Se exigido pela entidade, apresentar os dados curriculares do autor em
%     folha ou página distinta após a folha de rosto.''
% ---

% ---
% inserir lista de ilustrações
% ---
\pdfbookmark[0]{\listfigurename}{lof}
%\listoffigures*
\cleardoublepage
% ---

% ---
% inserir lista de tabelas
% ---
\pdfbookmark[0]{\listtablename}{lot}
%\listoftables*
\cleardoublepage
% ---

% ---
% inserir lista de abreviaturas e siglas
% ---
%\begin{siglas}
 % \item[Fig.] Area of the $i^{th}$ component
  %\item[456] Isto é um número
  %\item[123] Isto é outro número
  %\item[lauro cesar] este é o meu nome
%\end{siglas}
% ---

% ---
% inserir lista de símbolos
% ---
%\begin{simbolos}
%  \item[$ \Gamma $] Letra grega Gama
%  \item[$ \Lambda $] Lambda
%  \item[$ \zeta $] Letra grega minúscula zeta
%  \item[$ \in $] Pertence
%\end{simbolos}
% ---

% ---
% inserir o sumario
% ---
\pdfbookmark[0]{\contentsname}{toc}
\tableofcontents*
\cleardoublepage
% ---


% ----------------------------------------------------------
% ELEMENTOS TEXTUAIS
% ----------------------------------------------------------
\textual

% ----------------------------------------------------------
% Introdução
% ----------------------------------------------------------
\chapter*[Tema]{Tema}
\addcontentsline{toc}{chapter}{Introdução}
O tema de nossa pesquisa é a o que ocasiona as pessoas, em especial as mulheres, n\~{a}o se interessarem e, consequentemente n\~{a}o ingressarem na área de computaç\~{a}o e Técnologia da Informaç\~{a}o.

\nocite{nata}

\chapter{Problema de pesquisa}
A área de Técnologia da Informacao é uma área que esta sempre em crescimento por conta 
do avanço de nossa técnologia, porém em grande parte esta área é composta apenas por homens ou
apenas poucas mulheres trabalham neste ramo. O objetivo de nossa pesquisa é verificar o motivo pela qual
as mulheres não estão presentes nessa área, tanto no mercado de trabalho, mas principlamente nas academias ou universidades.


\chapter{Objetivos}
Nosso objetivo com essa pesquisa é poder explicar o motivo dessa imprecis\~{a}o para que os estudantes e profissionais da área de TI, mostrando que tal imprecis\~{a}o deverá ser levada em conta ao se desenvolver uma aplicaç\~{a}o que necessite de consistência na informaç\~{a}o  
\chapter{Justificativa}
Com nossa pesquisa poderemos saber o real motivo que ocasiona as mulheres n\~{a}o se interessarem pela área de Técnologia da Informaç\~{a}o e sabendo este motivo poderemos trabalhar em cima dos problemas que possivelmente ser\~{a}o encontrados para que possamos ter mais mulheres nesse ramo que se encontra em plena expans\~{a}o.
\chapter{Fundamentaç\~{a}o Teórica}
Teoria de ponto flutuante e padrão IEEE754 para auxilio da explicação.
Comumente tendemos a entender operações aritiméticas para o computador como algo simples e  rápido de se realizar. Números de natureza decimal podem apresentar problemas quando realizamos cálculos em um computador, o sistema de numerção decimal não é compreendido pela máquina,  mas sim o binário, dessa maneira  para que o computador possa fazer cálculos se faz necessária a transformação de decimal para a linguagem de maquina. Quando tratamos de números fracionarios devemos passa-lo para ponto flutuante, para que então o computador possa realizar cálculos.  Dessa maneira faz-se necessário enteder não somente como funciona a notação em ponto flutuante, mas também a normalização IEEE754, pois nos ajuda a explicar a questão da imprecisão como podemos observar no artigo “Padrão IEEE 754 para Aritmética Binária de Ponto Flutuante” \cite{viana1999padrao}.
 
 Arquitetura de computadores para melhor entendimento das grandezas utilizadas.
A arquitetura de computadores é fundamental para o entendimento do problema com imprecisão, uma vez que a uma maquina 32 bits processa menor extensão de casas decimais do que arquiteturas 64 bits, dessa maneira é possível entendender claramente a lógica de funcionamento da maquina quando realiza operações aritiméticas com números extensos. Podemos tomar maiores conhecimentos em “Arquitetura de computadores.” \cite{juniorarquitetura} .
\chapter{Metodologia de Pesquisa}
Para que nossa pesquisa se concretize, desfrutaremos de livros e artigos para nossas referências teóricas, também utilizamos formulários online para os questionários que foram distribuídos nos grupos de sites de relacionamento onde se encontravam um agrupamento de pessoas que atualmente s\~{a}o estudantes de cursos diferentes, para podermos saber também o que pessoas de outras áreas acham da área de Técnologia da Informaç\~{a}o.
O questionário consiste em 5 perguntas simples direcionada para as mulheres.

\chapter{Desenvolvimento}

A fim de obtermos a resposta para nosso questionamento inicial, elaboramos um questionário, para mulheres que est\~{a}o cursano ensino superior, ou que já tenha cursado. O questionário foi disponibilizado em um grupo de uma universidade em uma rede social onde alunos se unem para discutir assuntos relacionados à vida universitária e afins. O questionário se baseava em 5 perguntas simples. Colaboraram com a pesquisa 25 mulheres de diferentes áreas de atuaç\~{a}o.

Os resultados apontaram que 60\% das participantes s\~{a}o do segmento de humanas, 24\% eram de T.I. e os 16\% restantes eram do segmento de biológicas.
 	
% ----------------------------------------------------------
% Capitulo de textual  
% ----------------------------------------------------------


%\index{elementos textuais}A norma ABNT NBR 15287:2011, p. 5, apresenta a seguinte orientação quanto aos elementos textuais:

%\begin{citacao}
%O texto deve ser constituído de uma parte introdutória, na qual devem ser expostos o tema do projeto, o problema a ser abordado, a(s) hipótese(s), quando couber(em), bem como o(s) objetivo(s) a ser(em) atingido(s) e a(s) justificativa(s). É necessário que sejam indicados o referencial teórico que o embasa, a metodologia a ser utilizada, assim como os recursos e o cronograma necessários à sua consecução.
%\end{citacao}

%\chapter{Desenvolvimento}
%Uma situaç\~{a}o que ocorre muito na computaç\~{a}o é o arredondamento de números extensos, causando assim uma imprecis\~{a}o nos dados que nos s\~{a}o passados, formando assim, o objeto de nossa pesquisa, nosso objetivo é descobrir e explicar o porque isso ocorre e, para descobrir utilizaremos de livros e artigos específicos da área de arquitetura de computadores, e para provar as teorias, usaremos dos sistemas operacionais Linux e Windows, utilizando as linguagens de programaç\~{a}o C e Python, para que sejamos capazes de dizer se a inconsistência de dados ocorre apenas entre sistemas operacionais ou até mesmo entre linguagens de programaç\~{a}o. Com essa pesquisa poderemos auxiliar os estudantes da área de Tecnologia da Informaç\~{a}o a compreender o motivo da imprecis\~{a}o e também poderemos dizer a partir de que momento isso pode prejudicar o usuário.

% ----------------------------------------------------------
% Capitulo com exemplos de comandos inseridos de arquivo externo 
% ----------------------------------------------------------

\include{abntex2-modelo-include-comandos}

% ---
% Finaliza a parte no bookmark do PDF
% para que se inicie o bookmark na raiz
% e adiciona espaço de parte no Sumário
% ---
\phantompart

% ---
% Conclusão
% ---
%\chapter*[Considerações finais]{Considerações finais}
%\addcontentsline{toc}{chapter}{Considerações finais}

%\lipsum[31-32]

% ----------------------------------------------------------
% ELEMENTOS PÓS-TEXTUAIS
% ----------------------------------------------------------
\postextual

% ----------------------------------------------------------
% Referências bibliográficas
% ----------------------------------------------------------
\bibliography{abntex2-modelo-references}

% ----------------------------------------------------------
% Glossário
% ----------------------------------------------------------
%
% Consulte o manual da classe abntex2 para orientações sobre o glossário.
%
%\glossary

% ----------------------------------------------------------
% Apêndices
% ----------------------------------------------------------

% ---
% Inicia os apêndices
% ---
%\begin{apendicesenv}

% Imprime uma página indicando o início dos apêndices
%\partapendices

% ----------------------------------------------------------
%\chapter{Quisque libero justo}
% ----------------------------------------------------------

%\lipsum[50]

% ----------------------------------------------------------
%\chapter{Nullam elementum urna vel imperdiet sodales elit ipsum pharetra ligula
%ac pretium ante justo a nulla curabitur tristique arcu eu metus}
% ----------------------------------------------------------
%\lipsum[55-57]

%\end{apendicesenv}
% ---


% ----------------------------------------------------------
% Anexos
% ----------------------------------------------------------

% ---
% Inicia os anexos
% ---
%\begin{anexosenv}

% Imprime uma página indicando o início dos anexos
%\partanexos

% ---
%\chapter{Morbi ultrices rutrum lorem.}
% ---
%\lipsum[30]

% ---
%\chapter{Cras non urna sed feugiat cum sociis natoque penatibus et magnis dis
%parturient montes nascetur ridiculus mus}
% ---

%\lipsum[31]

% ---
%\chapter{Fusce facilisis lacinia dui}
% ---

%\lipsum[32]

%\end{anexosenv}

%---------------------------------------------------------------------
% INDICE REMISSIVO
%---------------------------------------------------------------------

\phantompart

\printindex


\end{document}
\grid
